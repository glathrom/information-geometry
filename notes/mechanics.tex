
\section{Mechanics}

Suppose we have a manifold $Q$ with local coordinates $(q^1,q^2,\ldots,q^n)$.
That is, $Q$ is a collection of charts, $\left\{ U_\alpha \right\}$, which is an
open cover of $Q$.  For each chart $U_\alpha$ we have a diffeomorphism
\begin{equation*} 
    \phi_\alpha:U_\alpha \ra \phi_\alpha(U_\alpha) \subseteq \bb
    R^n, 
\end{equation*} 
where, together with the natural projection functions
    from the product space $\bb R^n$ onto the individual coordinates, $\pi^i
    :\bb R^n \ra \bb R$,  we obtain the coordinate functions $q^i = \pi^i \circ
    \phi_\alpha$.  This gives rise to what we have referred to as the local
    coordinates above for the chart $U_\alpha$.  A \textbf{Lagrangian} is simply
    a function from the tangent bundle into the real numbers, $L:TQ \ra \bb R$.

