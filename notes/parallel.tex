\section{Parallel Transport}

Suppose we are given a manifold $M$ with a coordinate chart
\begin{equation*}
    x:U \subseteq \mathbb{R}^n \ra M.
\end{equation*}
The tangent space at a point $p \in M$ is a vector spanned by the vectors
\begin{equation*}
    \partial_i = \dfrac{\partial x}{\partial u^i}\biggl\lvert_p
\end{equation*}
or the column space of the Jacobian matrix, $J$,  evaluated at the point $p$, denoted $T_p(M)$.

We have $n$ of the vectors $\pd_i$ which may live in some unknown dimensional vector space in which the manifold is embedded.
In order to simplify things we define a metric tensor $g = J^tJ$ which gives a linear transformation 
\begin{equation*}
    g:T_p(M) \times T_p(M) \ra \mb R.
\end{equation*}
This means that our metric tensor can actually be viewed as an element 
\begin{equation*}
    g = g_{ij} dx^i\otimes dx^j \in T^\ast_p(M) \otimes T^\ast_p(M).
\end{equation*}
For tangent vectors $v^k\pd_k$ and $w^l\pd_l$ we have
\begin{align*}
    g(v,w) &= g_{ij}dx^i\otimes dx^j(v^k\pd_k, w^l\pd_l) \\
    &= g_{ij}v^kw^ldx^i(\pd_k)dx^j(\pd_l) \\
    &= g_{ij}v^kw^l \delta^i_k\delta^j_l\\
    &= g_{ij}v^iw^j = \inner{v|w}_g.
\end{align*}

Because each tangent space is attached at a point to the manifold, there is no natural way to move a vector from one tangent space to another.
The mechanism which does this is found in a natural way and is called a \textbf{parallel transport}.

If we are in the space $\mb R^m$ then we simply move the tangent vector from the point $p$ to the point $q$.
The problem with this is that the tangent space, has moved as we have gone from one point to another.
Therefore, what we need in order to move one tangent space to another is a projection onto the tangent space at $q$.
This will give us a linear map $T_p(M) \ra T_q(M)$.  
\begin{defn}
    A linear operator $P:V \ra V$ is a \textbf{projection} if $P^2 = P$.  
\end{defn}
If $M$ were an $n$-sphere, then for any point $p = (p^1,p^2,\ldots,p^n)$ then the point diametrically opposed to $p$, say $q = (-p^1,-p^2,\ldots,-p^n)$ will have parallel tangent spaces.

If a tangent vector happens to be in both tangent spaces, then we would want the image under the projection to be the identity map.

