\documentclass[12pt,letterpaper]{article}
\title{Information Geometry}
\author{G H Lathrom}

\usepackage{mystyle}
\usepackage[margin=1in,includehead]{geometry}
\usepackage{wasysym}
\usepackage{tensor}
\usepackage{hyperref}
\usepackage{tikz-cd}
\newcommand{\lspan}{\operatorname{span}}
\newcommand{\Ob}{\operatorname{Ob}}
\newcommand{\id}{\operatorname{id}}
\newcommand{\diag}{\operatorname{diag}}
\newcommand{\SO}{\operatorname{SO}}
\newcommand{\Oth}{\operatorname{O}}
\newcommand{\supp}{\operatorname{supp}}
\newcommand{\invlim}{\varprojlim}
\newcommand{\pow}[1]{\operatorname{Pow}\left( #1 \right)}
%\newcommand{\Prob}{\operatorname{Prob}}

\renewcommand{\theenumi}{\alph{enumi}}

\bibliographystyle{alpha}

\begin{document}
\maketitle


%%%%%%%%%%%%%%%%%%%%%%%%%%%%%%%%%
%%% Header Style
%%%%%%%%%%%%%%%%%%%%%%%%%%%%%%%%%

\pagestyle{fancy}
\fancyhf{}
\lhead{\slshape Information Geometry}
\chead{}
\rhead{\slshape GH Lathrom}
\fancyfoot[C]{\thepage}
%\renewcommand{\chaptermark}[1]{\markboth{\chaptername \ 
%\thechapter. \ #1}{}} 
\renewcommand{\headrulewidth}{.5pt}
%%%%%%%%%%%%%%%%%%%%%%%%%%%%%%%%%

\section{Background}

Information Geometry is a area which combines Differential Geometry,
Probability Theory and Information Theory into one discipline.  Because of
this, it is somewhat difficult to get a grasp on all of the concepts which go
into Information Theory.  By molding these three disciplines together we also
see connections between Physics and Machine Learning also with implications
for Artificial Intelligence.  Let's just see where all of these things take
us, while supplementing necessary background along the way.  One of the books
of interest will be \cite{dodson} along with a collection of papers available
\href{https://mega.nz/folder/5w0CDDzR\#EUMxMbyRJdNSKsqKIuEyFg}{here}.  We begin
with the connection between Information Geometry and Mechanics using
\cite{leok1} as the guide.  \begin{equation*} \tensor{\Gamma}{^{ij}_{k}} =
    \cdots \end{equation*}

\section{Mechanics}

Suppose we have a manifold $Q$ with local coordinates $(q^1,q^2,\ldots,q^n)$.
That is, $Q$ is a collection of charts, $\left\{ U_\alpha \right\}$, which is an
open cover of $Q$.  For each chart $U_\alpha$ we have a diffeomorphism
\begin{equation*} \phi_\alpha:U_\alpha \ra \phi_\alpha(U_\alpha) \subseteq \bb
    R^n, \end{equation*} where, together with the natural projection functions
    from the product space $\bb R^n$ onto the individual coordinates, $\pi^i
    :\bb R^n \ra \bb R$,  we obtain the coordinate functions $q^i = \pi^i \circ
    \phi_\alpha$.  This gives rise to what we have referred to as the local
    coordinates above for the chart $U_\alpha$.  A \textbf{Lagrangian} is simply
    a function from the tangent bundle into the real numbers, $L:TQ \ra \bb R$.

\section{Cohomology}

Beginning the discussion of Cohomology we have the notion of a singular chain
complex.  
\begin{equation*} 
    \begin{tikzcd} 
        \cdots \ar[r] & C_{i+1} \ar[r,"\partial_{i+1}"] & C_i 
        \ar[r, "\partial_i"] & C_{i-1} \ar[r] & \cdots
    \end{tikzcd} 
\end{equation*} 
If we take vectors $u_0, u_1, \ldots, u_n \in \bb R^n$ which are affinely independent, 
that is $\left\{ u_i-u_0 \right\}_{i=1}^n$ are linearly independent, 
then the simplex is the set 
\begin{equation*} 
    C^n = \left\{  p^0u_0+p^1u_1+\cdots+p^nu_n ~\biggr |~ p^i \geq 0 \textup{ for all
    } i \textup{ and } \sum_{i} p^i = 1 \right\}.  
\end{equation*} 
One could also recognize the $n$ dimensional probability simplex, or the set of all
finite probability distributions on $n$ elements, as 
\begin{equation*}
    \Delta^n = \left\{ \mb p \in \bb R^n ~\biggr |~ p^i \geq 0 \textup{ for } 
    i=0,1,\ldots,{n-1} \textup{ and } \sum_i p^i = 1 \right\}
\end{equation*} 
so that the simplex, $C^n$ may be recognized as the set of all possible expected-values 
with respect to the probability distributions
\begin{equation*} 
    C^n = \left\{ E[U|\mb p] = \braket{U}_{\mb p} ~|~ \mb p
        \in \Delta^{n+1}\right\} 
\end{equation*} 
where $U$ is a random variable on the set $\left\{ u_0, u_1, \ldots, u_n \right\}$ 
where $\prob(U=u_i ~|~ \mb p) = p^i$.




\bibliography{thebibliography}

\end{document}
