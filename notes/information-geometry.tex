\documentclass[12pt,letterpaper]{article}
\title{Information Geometry}
\author{G H Lathrom}

\usepackage{mystyle}
\usepackage[margin=1in,includehead]{geometry}
\usepackage{wasysym}
\usepackage{tensor}
\usepackage{hyperref}

\newcommand{\lspan}{\operatorname{span}}
\newcommand{\Ob}{\operatorname{Ob}}
\newcommand{\id}{\operatorname{id}}
\newcommand{\diag}{\operatorname{diag}}
\newcommand{\SO}{\operatorname{SO}}
\newcommand{\Oth}{\operatorname{O}}
\newcommand{\supp}{\operatorname{supp}}
\newcommand{\invlim}{\varprojlim}
\newcommand{\pow}[1]{\operatorname{Pow}\left( #1 \right)}
%\newcommand{\Prob}{\operatorname{Prob}}

\renewcommand{\theenumi}{\alph{enumi}}

\bibliographystyle{alpha}

\begin{document}
\maketitle


%%%%%%%%%%%%%%%%%%%%%%%%%%%%%%%%%
%%% Header Style
%%%%%%%%%%%%%%%%%%%%%%%%%%%%%%%%%

\pagestyle{fancy}
\fancyhf{}
\lhead{\slshape Information Geometry}
\chead{}
\rhead{\slshape GH Lathrom}
\fancyfoot[C]{\thepage}
%\renewcommand{\chaptermark}[1]{\markboth{\chaptername \ 
%\thechapter. \ #1}{}} 
\renewcommand{\headrulewidth}{.5pt}
%%%%%%%%%%%%%%%%%%%%%%%%%%%%%%%%%

\section{Background}

Information Geometry is a area which combines Differential Geometry, Probability Theory and Information Theory into one discipline.  
Because of this, it is somewhat difficult to get a grasp on all of the concepts which go into Information Theory.  
By molding these three disciplines together we also see connections between Physics and Machine Learning also with implications for Artificial Intelligence.
Let's just see where all of these things take us, while supplementing necessary background along the way.
One of the books of interest will be \cite{dodson} along with a collection of papers available \href{https://mega.nz/folder/5w0CDDzR\#EUMxMbyRJdNSKsqKIuEyFg}{here}.

\begin{equation*}
    \tensor{\Gamma}{^{ij}_{k}} = \cdots
\end{equation*}


\bibliography{thebibliography}

\end{document}
